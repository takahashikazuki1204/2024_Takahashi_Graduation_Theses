\newpage
\section*{概要}
McKibben 型人工筋肉 (McKibben Pneumatic actuator) は圧縮空気を印加することで収縮 し,自身の軸方向への張力を発生させるアクチュエータである.
近年では直径数 mmの細径MPAが注目されており,小さな筋肉のみならず集積化によって複雑な筋肉の再現が可能であることから筋骨格系ロボットや生物模倣ロボットへの
応用が盛んである.現在市販されている細径MPAは直径が3 mm程度のものが最も径が小さいが,一方でさらなる高集積化や小動物型や昆虫型などの非常に小型なロボットへ
の応用を考える上では,さらに細径な人工筋肉の開発が求められている.そこで本研究では比較的構造が簡単な軸方向繊維強化型人工筋肉に着目し,さらに細径な超細径空気圧人工筋肉
の開発を目指す.それに向けて本論文では液体ラテックスを用いた風船作製のノウハウを基に,内径5 mm,3 mmの細径な軸方向繊維強化型人工筋肉の作製方法の開発を行った.
さらに収縮率や発揮張力を実験を通じて確認した.

\newpage
\section*{Abstaract}
The McKibben pneumatic actuator (MPA) is an actuator that contracts when compressed air is applied to it, generating tension in its own axial direction.
Recently, thin MPAs with a diameter of a few millimeters have been attracting attention, and their application to musculoskeletal and bio-mimetic robots has been popular because they can reproduce not only small muscles but also complex muscles by integration.
The application of MPAs to musculoskeletal and biomimetic robots is flourishing. The smallest diameter MPAs currently available on the market are those with a diameter of about 3 mm.
On the other hand, the development of even smaller diameter artificial muscles is required for further integration and application to very small robots such as small animals and insects. In this study, we focus on axial fiber-reinforced artificial muscles, which have a relatively simple structure, and aim to develop even thinner ultra-thin pneumatic artificial muscles.
This paper describes the development of a liquid lathe. In this paper, we develop a method to fabricate axial fiber-reinforced artificial muscles with inner diameters of 5 mm and 3 mm, based on our expertise in balloon fabrication using liquid latex.
The shrinkage rate and exerted tension were confirmed through experiments.

