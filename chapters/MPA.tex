\newpage
\section{細径MPAの開発}
\subsection{風船作製}
中西研究室では液体ゴム(前加硫ラッテクス)を用いて細径MPAを作製したことがないので,研究の第1歩として風船作製を行った.
\subsubsection{作製手順}
図3に作製に必要な物品,図4に作製手順,図5に作製した風船を示す.必要な物品は以下の通りである.
\begin{itemize}
    \item 鉄棒(内径5 mm,3 mm)
    \item REGITEX 液体ゴム(前加硫ラッテクス) メーカー:有限会社 ハイラテック
    \item PC-518用 凝固液
    \item ドライヤー Panasonic EH-Ne13
  \end{itemize}
  以下,作製手順である.
\begin{enumerate}
    \item まず初めに鉄棒を凝固液に約5秒浸して取り出す
    \item 取り出した鉄棒を凝固液の水滴がなくなるまでドライヤーで乾かす(水滴が残っているとゴムがダマになってしまいゴム厚に偏りが生じ,破裂が起きやすくなるのでよく乾かす)
    \item 鉄棒を液体ゴムに約10秒浸して取り出す(容器に鉄棒が触れるとゴムの外膜が剥がれるので注意する)
    \item 凝固液に約5秒浸して取り出す
    \item 取り出した鉄棒をゴム膜の外側のいろが白色から肌色になるまでドライヤーで乾かす
    \item 3時間程部屋で乾かしたら鉄棒からゴム膜をとる(部屋で放置しすぎるとゴムが硬くなりすぎて鉄棒から取り外すときに割れたり,穴が空きやすくなるので注意する)
\end{enumerate}
以上が本研究で用いる風船の作製手順である.
\begin{figure}[!b]
  \centering  % 図全体を中央に配置
  \includegraphics[scale=0.3]{pic/kigu.PNG}
  \caption{使用器具}
\end{figure}
\begin{figure}[!t]
  \centering  % 図全体を中央に配置
  \includegraphics[scale=0.3]{pic/tezyun.PNG}
  \caption{風船の作製手順}
\end{figure}
\begin{figure}[!t]
  \centering  % 図全体を中央に配置
  \includegraphics[scale=0.3]{pic/balloon.jpg}
  \caption{作製した風船}
\end{figure}


\subsubsection{問題点の改善}
上記の作製方法で風船を作製するとゴム膜が硬すぎる,ゴム膜の上下で厚さにムラができる問題が生じた.この問題を解決するために以下の方法を行った.
まずゴム膜が硬すぎて膨張させることにかかる印加圧力が多く破裂が起きやすい問題に対して,ゴムと水の比率を変化させることで解決を行った.
初めはREGITEX 液体ゴム(前加硫ラッテクス)の原液のみで作製を行っていた.原液で作製を行うとゴム膜が硬すぎて図5のように破裂してしまった。
そこで実際に売られている風船と同じ比率のゴム60:水40での作製,ゴム膜が形成されるギリギリの比率を調べるためにゴム40:水60での作製を行った。
ゴム60:水40の割合で作製したゴム風船は図6のように破裂をすることなく膨張することが確認できた.一方でゴム40:水60の割合で作製したゴム風船はギリギリゴム膜を形成することができるが空圧を印加するとすぐに膨張が始まり,図7のように
破裂が生じてしまった.次にゴムの上下で厚さに差が出ている問題に対して作製方法の変更で解決を行った.図3を見れば分かるように鉄棒を液体ゴムに浸す時に1回で行っており,上下で液体ゴムに浸っている時間に差が生じてしまっていた.
そこで上と下2回に分けることで解決をした.
作製手順を以下と図8に示す.
\begin{enumerate}
  \item まず初めに鉄棒を凝固液に約5秒浸して取り出す
  \item 取り出した鉄棒を凝固液の水滴がなくなるまでドライヤーで乾かす(水滴が残っているとゴムがダマになってしまいゴム厚に偏りが生じ,破裂が起きやすくなるのでよく乾かす)
  \item 鉄棒を液体ゴムに約5秒浸して取り出す(容器に鉄棒が触れるとゴムの外膜が剥がれるので注意する)
  \item ドライヤーで凝固液の水滴がなくなるまで乾かす
  \item 鉄棒を液体ゴムに約5秒浸して取り出す
  \item 凝固液に約5秒浸して取り出す
  \item 取り出した鉄棒をゴム膜の外側のいろが白色から肌色になるまでドライヤーで乾かす
  \item 3時間程部屋で乾かしたら鉄棒からゴム膜をとる(部屋で放置しすぎるとゴムが硬くなりすぎて鉄棒から取り外すときに割れたり,穴が空きやすくなるので注意する)
\end{enumerate}
\begin{figure}[!t]
  \centering  % 図全体を中央に配置
  \includegraphics[scale=0.3]{pic/tezyun2.PNG}
  \caption{風船の作製手順}
\end{figure}

\newpage
\subsection{内径5 mmの人工筋肉の作製}
超細径人工筋肉開発の開発に先立ち,まずは内径5 mmの細径空圧筋の作製を行った.作製するにあたって糸を止めるための器具を開発した.
開発した器具の写真を図9,10使用方法について図10に示す.開発した器具は図9のように鉄棒を刺す穴の周りに8つの穴が空いている.中央の穴は鉄棒より少し広い5.3 mm,糸を通す穴は1 mmで設計している.
使用方法としては図10のように鉄棒に器具をはめナイロン糸を通す仕組みになっている.
\begin{figure}[!b]
  \centering  % 図全体を中央に配置
  \includegraphics[scale=0.3]{pic/kigu2.jpg}
  \caption{作製した器具}
\end{figure}
\begin{figure}[!b]
  \centering  % 図全体を中央に配置
  \includegraphics[scale=0.3]{pic/tukau2.PNG}
  \caption{作製した器具}
\end{figure}

\begin{figure}[!b]
  \centering  % 図全体を中央に配置
  \includegraphics[scale=0.3]{pic/tukau.jpg}
  \caption{使用方法}
\end{figure}
