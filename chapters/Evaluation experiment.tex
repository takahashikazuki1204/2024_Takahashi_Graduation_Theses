\newpage
\section{評価実験}
\subsection{収縮率}
作製した内径5 mm,3 mmの人工筋肉の収縮率の測定を行った.
収縮率は
$$\frac{(収縮後の長さ)-(収縮前の長さ)}{収縮後の長さ}\times 100\\$$
で計算を行った.
図30に測定するにああたって必要な部品を示す.必要な物品は以下の通りである.
\begin{itemize}
    \item 作製した人工筋肉(内径5 mm,3 mm)
    \item オイルレス エアーコンプレッサー 39 L
\end{itemize}
測定方法を以下に示す.
\begin{enumerate}
    \item 膨張する前の人工筋肉の長さを測定する
    \item 人工筋肉をエアーコンプレッサーに取り付ける
    \item 人工筋肉の膨張を確認しながらノズルを回し,破裂しないギリギリまで膨張させる
    \item 膨張した状態を保ちその状態を測定する
\end{enumerate}
\subsubsection{内径5 mmの測定}
図31,32に今回作製した内径5 mmの人工筋肉を示す.
上記の方法で空圧を印加すると0.02Mpaが膨張の最大となった.
空圧印加前の長さは75 mm,空圧印加後の長さは68 mmで収縮率を求める式に代入すると
$$\frac{(68 mm)-(75 mm)}{68 mm}\times 100\\$$
となり収縮率が10.7 \%となることが確認できた.
\subsubsection{内径3 mmの測定}
図33,34に今回作製した内径3 mmの人工筋肉を示す.
上記の方法で空圧を印加すると0.06Mpaが膨張の最大となった.
空圧印加前の長さは54 mm,空圧印加後の長さは49 mmで収縮率を求める式に代入すると
$$\frac{(49 mm)-(54 mm)}{49 mm}\times 100\\$$
となり収縮率が9.3 \%となることが確認できた.



