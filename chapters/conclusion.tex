\newpage
\section{結言}
本研究では,超細径空気圧人工筋肉の開発を目的として,内径5 mmと内径3 mmの軸方向繊維強化型人工筋肉の開発し,評価実験を通じて収縮率,発揮張力を通じて超細径空気圧人工筋肉の適切な作製方法,断面について考察をした.\\
 その結果,細径な空圧筋においては,糸を内包するほどのゴム厚はむしろ膨張を妨げ,また必要な印加圧力が高まることで破裂を引き起こす結果となった.一方で作製方法5では,糸はゴム膜の中央には配置されておらず,糸と糸の間に薄いゴム
 膜を張るような構成となった.このことにより,膨張がスムーズに行われ,空圧筋が高い柔軟性を持ちながらも,必要な強度を確保することができた.このことから細径化していくには,
 
