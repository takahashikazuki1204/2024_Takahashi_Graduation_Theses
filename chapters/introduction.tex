\newpage
\setcounter{page}{1}
\section{緒言}
近年,ロボティクスや医療分野において,柔軟性を持つアクチュエータの重要性が増している\cite{Hauser}.
特に,従来の硬いアクチュエータに代わる柔軟な人工筋肉は,より自然な動作や安全性の向上が期待されており,様々な応用が提案されている\cite{21}.
人工筋肉の中でも,特に細径人工筋肉はパワーアシストスーツや筋骨格ロボットにおいて重要な役割を果たす\cite{11}.
細径化を進めると,さらなる高集積化や小動物型や昆虫型などの非常に小型なロボットへの応用ができると考えられる\cite{mi14071431}.
現在市販されている細径空圧筋はMcKibben型が主流であり,最も細いもので直径が3 mm程度である.
McKibben型の空圧筋はゴムチューブの周りを編組スリーブで覆った構造であるが,さらなる細径化にはこの編組スリーブを細径化する必要がある.
市販の編組スリーブにおいてはやはり直径3 mm程度が最も細く,また編組スリーブに作製には製紐機 (ブレーダー)という特殊な機械が必要であることから自作も困難である.
これらの背景から,McKibben方式で空圧筋をこれ以上細径化することは困難であると考えられる.

そこで本研究では軸方向繊維強化型人工筋肉に注目する.軸方向繊維強化型人工筋肉ではゴムチューブ内に拘束繊維を内包する構造になっており.空気圧が供給されると,内圧が半径方向にのみ伝達
され,軸方向への効率的な収縮を引き起こす仕組みになっている.さらに,チューブに外挿されたリングの数を調整することで,ゴムチューブの膨張を抑制しつつ必要な収縮力を発揮できる\cite{22}.McKibben型人工筋肉
はナイロンメッシュを細くしていかなければ細径化ができないが,軸方向繊維強化型人工筋肉はゴム被膜と拘束繊維からなるため,
風船などを作成する手順を参考に薄いゴム膜を製作し,細い糸を拘束繊維として用いればより細径な空圧筋を作成することができるのではないかと考えた.
そこで本研究では,構造が簡単である軸方向繊維強化型人工筋肉に注目し,従来よりもさらに細径な空圧筋の開発を目的とする.

本論文の構成は以下の通りである.まず2章では従来のMcKibben型空気圧人工筋肉について述べてから,今回使用する軸方向繊維強化型人工筋肉について説明する.次に3章では研究の第1歩として
行った風船作製について述べたあと,内径5 mmの人工筋肉の作製について説明する.4章では内径3 mmの人工筋肉の作製について説明する.5章では作製した人工筋肉の評価実験について述べたのち,6章で細径するにあたって適切な作製方法,糸の配置
について述べる.最後に結言ではまとめと今後の予定について述べる.
