\newpage
\setcounter{page}{1}
\section{緒言}
近年,ロボティクスや医療分野において,柔軟性を持つアクチュエータの重要性が増している.
特に,従来の硬いアクチュエータに代わる柔軟な人工筋肉は,より自然な動作や安全性の向上が期待されており,様々な応用が模索されている.
人工筋肉の中でも,細径人工筋はパワーアシストスーツや筋骨格ロボットにおいて重要な役割を果たす\cite{1}.
しかし,現状においては人工筋肉の細径化に課題があり,これらの解決に向けたさらなる研究が必要とされている.
中西研究室でもMcKibben型人工筋肉の細径化に成功しているが\cite{2},内径3 mm以下の人工筋肉の作成はスリーブとゴムチューブの作成が困難なためこれ以上の細径化は難しいと考えられる.
そこで本研究では,構造が簡単である軸方向繊維強化型人工筋肉\cite{3}を用いて,従来よりもさらに細径な空圧筋の開発を目的とする.