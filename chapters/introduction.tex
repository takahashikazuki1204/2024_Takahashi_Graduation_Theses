\newpage
\setcounter{page}{1}
\section{緒言}
近年,ロボティクスや医療分野において,柔軟性を持つアクチュエータの重要性が増している\cite{Hauser}.
特に,従来の硬いアクチュエータに代わる柔軟な人工筋肉は,より自然な動作や安全性の向上が期待されており,様々な応用が提案されてる.
人工筋肉の中でも,特に細径人工筋肉はパワーアシストスーツや筋骨格ロボットにおいて重要な役割を果たす\cite{11}.
細径化を進めると進めると,さらなる高集積化や小動物型や昆虫型などの非常に小型なロボットへの応用ができると考えられる\cite{mi14071431}.
しかし,現状においては人工筋肉の細径化には収縮性能などの課題があり,これらの解決に向けたさらなる研究が必要とされている.
また中西研究室でもMcKibben型人工筋肉の細径化に成功しているが\cite{22},内径3 mm以下のナイロンメッシュが市販されていないためこれ以上の細径化は難しいと考えられる.
そこで本研究では,構造が簡単である軸方向繊維強化型人工筋肉[4]に注目する.よって従来よりもさらに細径な空圧筋の開発を目的とする.

本論文の構成は以下の通りである.まず2章では従来のMicKibben型空気圧人工筋肉について述べてから,今回使用する軸方向繊維強化型人工筋肉について説明する.次に3章では研究の第1歩として
行った風船作製について述べたあと,内径5 mmの人工筋肉の作製について説明する.4章では内径3 mmの人工筋肉の作製について説明する.5章では作製した人工筋肉の評価実験について述べたのち,6章で細径するにあたって適切な作製方法,糸の配置
について述べる.最後に結言ではまとめと今後の予定について述べる.
