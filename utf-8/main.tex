\documentclass[11pt]{jarticle}

\usepackage[dvipdfmx]{graphicx}
\usepackage{sty/fancyheadings}
\usepackage{sty/sotsuron2009}
\usepackage{here}
\usepackage{ascmac}
\usepackage[subrefformat=parens]{subcaption}
\usepackage{here}
\usepackage{cite}
\usepackage{amsmath}
\pagestyle{empty}
\usepackage{bm}
\usepackage{comment}
\linesparpage{40}
%\makeatletter
%\renewcommand{\theequation}{
%\thesection.\arabic{equation}}
%\@addtoreset{equation}{section}
%\makeatother
\pagestyle{fancy}

\newcommand{\argmin}{\mathop{\rm arg~min}\limits}

\begin{document}

\begin{comment}
\title{
 \LARGE
 % 卒論和文タイトル
 \\ \\
 \large
 % 卒論英文タイトル
}

\author{
 研究者 高専 太郎\\
 指導教員 中西 大輔\\ \\
 松江工業高等専門学校\\
 電子制御工学科\\ \\
}

\date{平成29年2月13日}

\maketitle
\end{comment}

\thispagestyle{empty}
\newpage
\section*{概要}
McKibben 型人工筋肉 (McKibben Pneumatic actuator) は圧縮空気を印加することで収縮 し,自身の軸方向への張力を発生させるアクチュエータである.
近年では直径数 mmの細径MPAが注目されており,小さな筋肉のみならず集積化によって複雑な筋肉の再現が可能であることから筋骨格系ロボットや生物模倣ロボットへの
応用が盛んである.現在市販されている細径MPAは直径が3 mm程度のものが最も径が小さいが,一方でさらなる高集積化や小動物型や昆虫型などの非常に小型なロボットへ
の応用を考える上では,さらに細径な人工筋肉の開発が求められている.そこで本研究では比較的構造が簡単な軸方向繊維強化型人工筋肉に着目し,さらに細径な超細径空気圧人工筋肉
の開発を目指す.それに向けて本論文では液体ラテックスを用いた風船作製のノウハウを基に,内径5 mm,3 mmの細径な軸方向繊維強化型人工筋肉の作製方法の開発を行った.
さらに収縮率や発揮張力を実験を通じて確認した.

\newpage
\section*{Abstaract}
The McKibben pneumatic actuator (MPA) is an actuator that contracts when compressed air is applied to it, generating tension in its own axial direction.
Recently, thin MPAs with a diameter of a few millimeters have been attracting attention, and their application to musculoskeletal and bio-mimetic robots has been popular because they can reproduce not only small muscles but also complex muscles by integration.
The application of MPAs to musculoskeletal and biomimetic robots is flourishing. The smallest diameter MPAs currently available on the market are those with a diameter of about 3 mm.
On the other hand, the development of even smaller diameter artificial muscles is required for further integration and application to very small robots such as small animals and insects. In this study, we focus on axial fiber-reinforced artificial muscles, which have a relatively simple structure, and aim to develop even thinner ultra-thin pneumatic artificial muscles.
This paper describes the development of a liquid lathe. In this paper, we develop a method to fabricate axial fiber-reinforced artificial muscles with inner diameters of 5 mm and 3 mm, based on our expertise in balloon fabrication using liquid latex.
The shrinkage rate and exerted tension were confirmed through experiments.

        % 概要

\thispagestyle{empty}
\newpage
\tableofcontents
\thispagestyle{empty}

%章ごとに呼び出し

\newpage
\setcounter{page}{1}
\section{緒言}
近年,ロボティクスや医療分野において,柔軟性を持つアクチュエータの重要性が増している\cite{Hauser}.
特に,従来の硬いアクチュエータに代わる柔軟な人工筋肉は,より自然な動作や安全性の向上が期待されており,様々な応用が提案されてる.
人工筋肉の中でも,特に細径人工筋肉はパワーアシストスーツや筋骨格ロボットにおいて重要な役割を果たす\cite{11}.
細径化を進めると進めると,さらなる高集積化や小動物型や昆虫型などの非常に小型なロボットへの応用ができると考えられる\cite{mi14071431}.
しかし,現状においては人工筋肉の細径化には収縮性能などの課題があり,これらの解決に向けたさらなる研究が必要とされている.
また中西研究室でもMcKibben型人工筋肉の細径化に成功しているが\cite{22},内径3 mm以下のナイロンメッシュが市販されていないためこれ以上の細径化は難しいと考えられる.
そこで本研究では,構造が簡単である軸方向繊維強化型人工筋肉[4]に注目する.よって従来よりもさらに細径な空圧筋の開発を目的とする.

本論文の構成は以下の通りである.まず2章では従来のMicKibben型空気圧人工筋肉について述べてから,今回使用する軸方向繊維強化型人工筋肉について説明する.次に3章では研究の第1歩として
行った風船作製について述べたあと,内径5 mmの人工筋肉の作製について説明する.4章では内径3 mmの人工筋肉の作製について説明する.5章では作製した人工筋肉の評価実験について述べたのち,6章で細径するにあたって適切な作製方法,糸の配置
について述べる.最後に結言ではまとめと今後の予定について述べる.
    % 緒言
\newpage
\section{空気圧人工筋肉}
\subsection{McKibben型空気圧人工筋肉アクチュエータ(MPA)}
MPAはシリコンゴムチューブをナイロンメッシュで覆うことで構成されており(図\ref{fig:MPA}\subref{fig:Structure}),両端に栓をするシンプルな構造である.
これに圧縮した空気を印加することでシリコンゴムチューブが膨張しメッシュによる自身の軸方向への張力が発生するアクチュエータである(図\ref{fig:MPA}\subref{fig:move}).
高出力かつ素材自体も軽量で,物理的柔軟性による高い弾性力を持つという利点があり,筋肉の代用として生物を模したロボットやリハビリなどに用いられる.
しかし図1に示すような従来の直径が数10 mmのMPAは,さらなる高集積化や小動物型や昆虫型などの非常に小型なロボットへの応用を考慮した場合,そのサイズは大きすぎる
%%%%%%%%%%%%%%%%%%%%%%%%%%%%%%%%%%%%%%%%%%%%%%%%%%%%%%%%%
\begin{figure}[b]
    %
    \begin{minipage}{0.49\columnwidth}
      \vspace{4mm}
      \centering
      \includegraphics[scale=1]{pic/MPA_kousei.png}
      \vspace{3mm}
      \subcaption{MPA断面図}
      \label{fig:Structure}
    \end{minipage}
    %
    \begin{minipage}{0.49\columnwidth}
      \vspace{25mm}
      \centering
      \includegraphics[scale=.8]{pic/MPA_dousa.png}
      \subcaption{MPA外観および動作の様子}
      \label{fig:move}
    \end{minipage}
    %
    \caption{McKibben型空気圧人工筋(MPA)の構成および外観\cite{中西大輔2020}}
    \label{fig:MPA}
  \end{figure}.
  \begin{figure}[!b]
    \centering  % 図全体を中央に配置
    \includegraphics[scale=0.7]{pic/A.PNG}
    \caption{軸方向繊維強化型人工筋肉の仕組み\cite{4}}
  \end{figure}
  
\subsection{軸方向繊維強化型人工筋肉}
本研究で開発する超細径空圧筋のベースとなる,軸方向繊維強化型人工筋肉の動作原理を図1に示す.
この人工筋肉は,ゴムチューブ内に拘束繊維を内包する構造で,拘束繊維とゴムチューブとの摩耗を抑え長寿命を実現する.
空気圧が供給されると,内圧が半径方向にのみ伝達され,軸方向への効率的な収縮を引き起こす.
さらに,チューブに外挿されたリングの数を調整することで,ゴムチューブの膨張を抑制しつつ必要な収縮力を発揮できる\cite{3}.

            % 本文
% 必要に応じて章を増やす,またファイル名もsec2, sec3である必要はない
% このmain.texが置いてあるディレクトリ内にある「chapters」フォルダ以下に
% 自分がわかりやすい名前のtexファイルを作成し,\include{chapters/*****}で
% 呼び出せば良い
\newpage
\section{結言}
本研究では,超細径空気圧人工筋肉の開発を目的として,内径5 mmと内径3 mmの軸方向繊維強化型人工筋肉の開発し,評価実験を通じて収縮率,発揮張力を通じて超細径空気圧人工筋肉の適切な作製方法,断面について考察をした.\\
 その結果,細径な空圧筋においては,糸を内包するほどのゴム厚はむしろ膨張を妨げ,また必要な印加圧力が高まることで破裂を引き起こす結果となった.一方で作製方法5では,糸はゴム膜の中央には配置されておらず,糸と糸の間に薄いゴム
 膜を張るような構成となった.このことにより,膨張がスムーズに行われ,空圧筋が高い柔軟性を持ちながらも,必要な強度を確保することができた.このことから細径化していくには,
 
      % 結言
\newpage
\section{謝辞}
本研究を進めるにあたり,数多くの助言,提案,活発な議論をしていただいた中西大輔
先生に心から感謝申し上げます.また,様々なご助言やご協力をいただきました中西研究
室の皆様に,心から感謝いたします % 謝辞
% できればbibtexを使ってください

\newpage
\section*{参考文献}
\bibliographystyle{junsrt}
\bibliography{sanko.bib}


    % 参考文献

\end{document}
